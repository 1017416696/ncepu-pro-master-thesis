%-------------------------------------------------
% FileName: appendix.tex
% Author: Caoxiaofan (120222327024@ncepu.edu.cn)
% Version: 0.1
% Date: 2020-05-12
% Description: 攻读硕士学位期间发表的论文及其它成果
% Others: 如果没有内容,就在main.tex中注释掉
% History: origin
%------------------------------------------------- 


% 以下不用改动-------------------------------------
% 断页
% \clearpage



% hyperef 精确定位
% 设置一个anchor,主要针对\addcontentsline
% 防止目录,书签等指向错误位置 
% \phantomsection
% 增加到目录,与chapter同级别
% \addcontentsline{toc}{chapter}{\appendixname} 
% \chapter*{} 表示不编号,不生成目录
% \markboth 用于页眉
% \chapter*{\appendixname \markboth{\appendixname}{}} 

% 断页
\clearpage

% hyperef 精确定位
\phantomsection
% 增加到目录,与chapter同级别
\addcontentsline{toc}{chapter}{攻读硕士学位期间发表的论文及其它成果}
% \chapter*{} 表示不编号,不生成目录
\chapter*{攻读硕士学位期间发表的论文及其它成果 \markboth{攻读硕士学位期间发表的论文及其它成果}{}}

% 在此处添加发表的论文及成果列表
% 例如:

% 设置字体、字号和行距
{\songti\zihao{-4}\linespread{1.25}\selectfont
\setlength{\parindent}{0em}  % 将段落缩进设为 0

% 定义新的环境,使用您指定的格式
\newenvironment{achievement}
{\begin{enumerate}[label={[\arabic*]}, align=left, leftmargin=2em, itemindent=-1.5em, labelsep=0.5em]}
{\end{enumerate}}

% 设置字体、字号和行距
{\songti\zihao{-4}\linespread{1.25}\selectfont
\setlength{\parindent}{0em}  % 将段落缩进设为 0

\vspace{0.5\baselineskip}  % 段前 0.5 行
\textbf{(一)发表的学术论文}
\vspace{0.5\baselineskip}  % 段后 0.5 行

\begin{achievement}
\item ×××, ×××. 部多孔质气体静压轴向轴承静态特性的数值求解[J]. 摩擦学学报, 2007, 38(12)68~72(EI 收录号: 071510544816)
\item 
\end{achievement}

\vspace{1em}



\textbf{(二)申请及已获得的专利(无专利时此项不必列出)}
\vspace{0.5\baselineskip}

\begin{achievement}
\item ×××, ×××. 一种温热外敷药制备方案: 中国, 88105607.3 [P]. 1989-07-26.
\end{achievement}

\vspace{1em}


\textbf{(三)获得的科技奖励(无获奖时此项不必列出)}
\vspace{0.5\baselineskip}

\begin{achievement}
\item ×××, ×××, ××静载下预应力混凝土房屋结构设计统一理论. 黑龙江省科学技术二等奖, 2007.
\end{achievement}

\vspace{2em}

注: \,如已发表的学术论文被 EI 或 SCI 收录,请标明收录号及 SCI 论文的影响因子;对已接收但尚未发表出来的学术论文,请注明是否 EI 或 SCI 刊源。

}