%-------------------------------------------------
% FileName: title-ch.tex
% Author: Caoxiaofan (120222327024@ncepu.edu.cn)
% Description: 中文扉页
% Version: 0.1
% Date: 2024-11-12
% History: origin
%-------------------------------------------------

\thispagestyle{empty}

% 使用 tabular 环境来实现两端对齐
\noindent
\begin{tabular*}{\textwidth}{@{\hspace{0pt}}l@{\extracolsep{\fill}}l@{\hspace{0pt}}}
    {\songti \zihao{-4} 国内图书分类号:××××} & {\songti \zihao{-4} 学校代码:10054} \\
    {\songti \zihao{-4} 国际图书分类号:××××} & {\songti \zihao{-4} 密\hspace{2em}级:公开}
\end{tabular*}
    
    \vspace{3cm}

\begin{center}

    {\songti \zihao{-2} \bfseries 专业硕士学位论文}
    
    \vspace{2cm}
    
    {\heiti \zihao{2} \deftitle}
    
    \vspace{6cm}
    
    % 设置行距为1.5倍
    \renewcommand{\arraystretch}{1.5}
    \hspace*{-0.5cm}  % 添加负值的水平空间来向左移动表格
        \begin{tabular}{>{\heiti\zihao{4}}p{4cm}@{\hspace{-1.4em}}>{\songti\zihao{4}}l}
        \makebox[3cm][s]{硕士研究生}:\space & \defstudent \\
        \makebox[3cm][s]{导\hspace{4em}师}:\space & \defadvisor xxx(副)教授 \\
        \makebox[3cm][s]{企业导师}:\space & \defcoadvisor\space 高级工程师 \\
        \makebox[3cm][s]{申请学位}:\space & **硕士 \\
        \makebox[3cm][s]{专业领域}:\space & \defmajor \\
        \makebox[3cm][s]{学习方式}:\space & 全日制/非全日制 \\
        \makebox[3cm][s]{所在学院}:\space & 电气与电子工程学院 \\
        \makebox[3cm][s]{答辩日期}:\space & 2024年6月 \\
        \makebox[3cm][s]{授予学位单位}:\space & 华北电力大学
    \end{tabular}
\end{center}