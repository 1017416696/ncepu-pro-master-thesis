%-------------------------------------------------
% FileName: chapt-1.tex
% Author: Caoxiaofan (120222327024@ncepu.edu.cn)
% Version: 0.1
% Date: 2024-11-12
% Description: 第1章
% Others: 
% History: origin
%------------------------------------------------- 

% 断页
\clearpage
% 页码从1开始计数
\setcounter{page}{1} 
% 阿拉伯数字显示页码
\pagenumbering{arabic}

\chapter{绪论}

\section{课题背景及研究的目的和意义}
发展国防工业、微电子工业等尖端技术需要精密和超精密的仪器设备,精密仪器设备要求高速、……


% 引用参考文献 [1]
叔本华认为,我们可以提高自己对这世界的认识(当然是去认识叔本华所理解的那个世界),把自己的感情和欲望上升为全人类的感情和欲望,这样就可以消除个人的欲望\cite{chen2005laser,mittelbach2004latex}。

% 引用多篇参考文献 [1-3]
比如通过宗教的约束,比如通过立法的形式,遏制垄断企业(可怜的微软),遏制不正当\cite{arm}和不道德的竞争,给工人更多的福利\cite{mittelbach2004latex, arm,zhen2018leave}。

% 下面是一个有序列表的例子,默认编号
别人已经研究的工作包括:
\begin{enumerate}
	\item 古希腊的斯多葛学派就相信部分决定论。他们认为我们不能控制事物,但是可以控制我们自己对待生活的方式。所以这个学派提倡随遇而安的生活态度\cite{zhou2002nerualnet}。
	\item 斯宾诺莎是用类似于几何的逻辑一步步推出整个哲学体系的。这意味着,他相信世间万物之间都有着严格的逻辑关系。这必然也会导致决定论\cite{qi2020deeplearning}。
	\item 我们没必要也没能力去无限地怀疑世界\cite{partl2019short}。
\end{enumerate}

% 有序列表嵌套 定制编号
唐诗,宋词,元曲举例:
\begin{enumerate}
	\item 唐诗
	\begin{enumerate} 
		\item xxx
		\item xxx
	\end{enumerate}
	\item 宋词
	\begin{enumerate}
		\item xxxx
		\item xxx
	\end{enumerate}
	\item 元曲
	\begin{enumerate}
		\item xxx
		\item xxx
	\end{enumerate}
\end{enumerate}

\section{气体润滑轴承及其相关理论的发展概况}
气体轴承是利用气膜支撑负荷或减少摩擦的机械构件。……



% 下面演示怎么增加子标题
\subsection{气体润滑轴承的发展}
1828年,R.R.Willis\cite{arm}发表了一篇关于小孔节流平板中压力分布的文章,这是有记载的研究气体润滑的最早文献。

\subsection{气体润滑轴}

\subsection{气体润滑轴承}

\subsection{气体润滑轴承的}

\subsection{多孔质气体静压轴承的研究}
由于气体的压力低和可压缩性,……
\subsubsection{多孔质静压轴承的分类}
\subsubsection{多孔质材料特性的研究}
材料的主要特点是具有一定的…

% \subsection{目的意义2}
% \subsection{目的意义3}
% \subsection{目的意义4}

% \section{论文主要工作}
% 介绍本研究课题的来源及主要研究内容。

% 本作品分工如下,虚若无同学实现:
% % 下面是一个无序列表的例子
% \begin{itemize}
% 	\item 系统架构设计;
% 	\item 功能模块的设计与实现;
% 	\item web端的编程与实现;
% 	\item 数据库设计。
% \end{itemize}

% 欧阳潇潇同学实现:
% \begin{itemize}
% 	\item 微信小程序的设计与实现;
% 	\item 微信端接口的实现;
% 	\item 数据库设计。
% \end{itemize}

% \section{论文组织结构}

% 第1章介绍了考研教室预约系统的课题背景,目的意义,组员分工,全文的组织结构。

% 第2章介绍了系统开发所涉及的相关技术。包括MySQl数据库,前后端分离。Spring Boot,Ajax,微信小程序开发。
 
% 第3章对考研教室预约系统做了详细的需求分析,详细介绍了系统在实际应
% 用中的功能需求,系统的业务分析,系统的用例分析,系统的功能分析和非功能性需求。

% 第4章考研教室预约系统的系统总体设计,详细介绍了系统的总体结构和系统模块的设计以及数据库的设计和E-R图。

% 第5章考研教室预约系统的系统的实现,介绍了系统实现的关键技术,以及WEB端和移动端各个功能模块的实现。

% 第6章总结和展望,总结了系统的开发工作,分析了系统目前存在的问题及系统需要进一步完善的地方。